\documentclass[a4paper,12pt]{article}
\usepackage[utf8]{inputenc}
\usepackage{geometry}
\usepackage[hidelinks]{hyperref}
\usepackage{titlesec}
\usepackage{tocloft}
\usepackage[german]{babel}
\usepackage{datetime}
\usepackage{graphicx} 
\usepackage{xcolor}
\usepackage{natbib} % Benötigt für BibTeX
\bibliographystyle{alpha} % Stil für BibTeX
\usepackage{setspace}
\onehalfspacing
\usepackage{enumitem}
\renewcommand{\arraystretch}{2}

\renewcommand{\dateseparator}{.} % Setzt das Datumsformat auf TT.MM.JJJJ
\geometry{a4paper, margin=2.5cm}

\begin{document}

%\maketitle
% Deckblatt
\begin{titlepage}
    \begin{flushleft}
        \includegraphics[width=5cm]{logo.png}
    \end{flushleft}
    
    \vfill
    
    \begin{center}
        \Huge \textbf{Qualitätssicherungsplan} \\[1.0cm]
        \Large NextGen Development \\[0.5cm]
        \textcolor{gray}{\rule{0.8\textwidth}{0.4pt}} \\[0.5cm]
        \textbf{Version 1.0} \\[0.5cm] %BITTE NACH ÄNDERUNGEN ANPASSEN!
        \large \today \\[1.5cm]
    \end{center}
    
    \vfill

    \noindent
    \begin{minipage}[t]{0.45\textwidth}
        \textbf{Teammitglieder:}\\
        Julian Lachenmaier\\
        Ayo Adeniyi\\ 
        Din Alomerovic\\ 
        Cedric Balzer\\ 
        Rebecca Niklaus\\
    \end{minipage}%
    \hspace{1cm} 
    \begin{minipage}[t]{0.45\textwidth} 
        \textbf{Verantwortlich für dieses Dokument:}\\
        Rebecca Niklaus (Qualitätssicherung)\\
    \end{minipage}

\end{titlepage}

\tableofcontents
\newpage

\section{Einleitung}
Dieser Qualitätssicherungsplan definiert die Maßahmen zur Sicherstellung der Qualität aller Ergebnisse des Projekts. Er beinhaltet eine klare Gliederung, die Verwendung von Templates und Versionierung sowie spezifische Qualitätsstandards.
\newpage

\section{Ziele}
Die Hauptziele der Qualitätssicherung sind:
\begin{itemize}
    \item Sicherstellung der Einhaltung von Standards und Best Practices
    \item Vermeidung und frühe Erkennung von Fehlern
    \item Sicherstellung einer einheitlichen und gut dokumentierten Codebasis
    \item Klare Struktur und Versionierung der Dokumentation
\end{itemize}
\newpage

\section{Projektmanagement}
Ziel des Projektmanagements ist, Risiken zu begrenzen und Projektziele unter Verwendung der verfügbaren Ressourcen zu erreichen. Das Endergebnis des Projekts soll den Anforderungen des Kunden entsprechen. Im folgenden Abschnitt werden Maßnahmen definiert, um dieses auf Seiten des Projektmanagements zu erreichen.

Um einen organisierten Projektablauf zu gewährleisten, werden häufig Methoden wie Scrum oder Kanban verwendet. Dieses Projekt orientiert sich an der Kanban-Vorgehensweise.

Kanban ist eine agile Methode. Das heißt, dass man die Arbeitsweise im Laufe des Projekts durch kleine Änderungen laufend verbessert \cite{dechange_agiles_2024}. Zur Visualisierung der Aufgaben des Teams wird ein Kanban-Board verwendet. Diese werden in drei Zustände unterteilt: To-Do, in Bearbeitung und vollständig. Für die Projektplanung wird das Tool \textbf{ClickUp} verwendet. 

    \subsection{Ressourcen-Planung} Für die Bearbeitung eines Projekts stehen begrenzte Ressourcen zur Verfügung. Im Rahmen dieses Projekts sind Personal- und zeitliche Ressourcen relevant. 
    
    Das Zeitmanagement stellt sicher, dass Abgabefristen eingehalten werden können \cite{helmold_projektmanagement_2023}. Jedem der fünf Projektmitglieder steht wöchentlich 10 Stunden zur Verfügung. Um diese möglichst sinnvoll zu nutzen, wird die Aufgabenverteilung auf die Fähigkeiten und Erfahrungen der Mitglieder angepasst. Die Zeiterfassung erfolgt in ClickUp.

    \subsection{Projektphasen und Meilensteine} Das Projekt wird in mehrere Phasen unterteilt, um eine strukturierte Vorgehensweise sicherzustellen. Jeder Phase sind klare Meilensteine zugewiesen, um den Fortschritt messbar zu machen. Diese sind im Projektplan festgelegt.
    
    Zu Beginn steht die Planungsphase, in der die Anforderungen und Ziele des Projekts definiert werden. In dieser Phase wird außerdem das Kanban-Board erstellt und erste Aufgaben zugewiesen. Anschließend folgt die Entwicklungsphase, in der die Kernfunktionen des Projekts umgesetzt werden. Während dieser Phase erfolgen regelmäßige Tests und interne Reviews, um den Fortschritt zu überprüfen.

    Nach der Entwicklung beginnt die Testphase, in der Funktionstests durchgeführt und identifizierte Fehler behoben werden. Diese Phase beinhaltet auch Feedback-Runden mit dem Team, um mögliche Optimierungen vorzunehmen. Abschließend folgt die Abschlussphase, in der letzte Qualitätskontrollen durchgeführt, die finale Dokumentation erstellt und das Projekt offiziell abgeschlossen wird. Der strukturierte Aufbau mit klaren Meilensteinen stellt sicher, dass das Projekt effizient durchgeführt und die gesetzten Ziele erreicht werden.

    \subsection{Kommunikationsmanagement} Grundvorraussetzung für einen erfolgreichen Projektablauf stellt die Zusammenarbeit und das Wohlbefinden der einzelnen Mitglieder dar. Sollten sich Mitglieder über- oder unterfordert fühlen, soll dies direkt kommuniziert werden können. Daher werden zwei Teams-Meetings á 15 Minuten pro Woche fest eingeplant, um mögliche Engpässe frühzeitig zu beheben. Außerdem besteht jederzeit die Möglichkeit, sich über Teams oder WhatsApp an sein Team zu wenden.

    \subsection{Risikomanagement} Bereits zu Beginn des Projekts werden mögliche Risiken, die im Laufe des Projekts zu Problemen führen können, analysiert. Im Bereich des Projektmanagements werden hierbei insbesondere Risiken genannt, die sich auf einen Ressourcen-Mangel beziehen. Daher wird eine Risikoanalyse durchgeführt, um präventive Maßnahmen vor Beginn des Projekts zu definieren und um auf mögliche Komplikationen vorbereitet zu sein.

    \subsection{Checkliste}
    Die folgende Checkliste fasst die wichtigsten Punkte zusammen, die während des Projekts im Bereich Projektmanagement kontinuierlich beachtet und umgesetzt werden sollten.
    \begin{itemize}[label=\textbf{X}]
        \item Aufgabenverteilung basiert auf den Erfahrungen der Mitglieder
        \begin{itemize}
            \item Stärken und Schwächen der Teammitglieder berücksichtigt
            \item Anpassung bei Bedarf während des Projekts
        \end{itemize}
        \item Projektphasen sind klar definiert
        \item Zwei wöchentliche Teams-Meetings á 15 Minuten eingeplant, werden bei Bedarf durchgeführt
        \begin{itemize}
            \item Dienstags: Wöchentlicher Check-in
            \item Donnerstags: Besprechung der Review
        \end{itemize}
        \item Präventive Risikomaßnahmen sind berücksichtigt
        \begin{itemize}
            \item Risikoanalyse vor Projektstart abgeschlossen
            \item Regelmäßige Überprüfung von Engpässen
        \end{itemize}
    \end{itemize}
    

\newpage

\section{Qualitätssicherungsmaßnahmen}
\subsection{Code-Qualität}
\begin{itemize}
    \item \textbf{Vier-Augen-Prinzip}: Jeder Code-Commit muss vor der Freigabe von mindestens einer weiteren Person überprüft werden.
    \item \textbf{Code-Conventions}: Einhaltung definierter Codierstandards, z.B. Formatierung, Benennungsregeln und Kommentierungsrichtlinien.
    \item \textbf{Pair-Programming}: Regelmäßige Anwendung zur Steigerung der Code-Qualität.
    \item \textbf{Automatisierte Tests}: Implementierung von Unit-Tests, Integrationstests und ggf. UI-Tests. \cite{helmold_projektmanagement_2023}

\end{itemize}

\subsection{Versionskontrolle}
\begin{itemize}
    \item Verwendung von \textbf{Git} zur Nachverfolgbarkeit von Änderungen.
    \item Nutzung eines definierten \textbf{Branching-Modells} (z.B. Git Flow).
    \item Reviews und Merge-Requests als Pflichtprozess vor der Zusammenführung in den Hauptbranch.
\end{itemize}

\subsection{Entwicklungsumgebung}
\begin{itemize}
    \item \textbf{Einheitliche Konfigurationen f\"ur IDEs}: Verwendung gemeinsamer Settings und Plugins.
    \item Definierte \textbf{Versionen} von Abhängigkeiten und Bibliotheken zur Sicherstellung der Reproduzierbarkeit.
    \item Bereitstellung eines Setup-Guides zur einfachen Einrichtung der Entwicklungsumgebung.
\end{itemize}
\newpage

\section{Dokumentationskonzept}
\subsection{Arten von Dokumenten}
Folgende Dokumente fallen w\"ahrend des Projekts an:
\begin{itemize}
    \item \textbf{Architekturdokumentation}: Struktur und technische Entscheidungen.
    \item \textbf{Anforderungsdokumentation}: Spezifikationen und Ziele.
    \item \textbf{Testdokumentation}: Teststrategien, Testfälle und Testergebnisse.
    \item \textbf{Entwicklerdokumentation}: Code- und API-Dokumentation.
    \item \textbf{Nutzerdokumentation}: Anleitungen und Manuals.
\end{itemize}

\subsection{Qualitätsstandards}
\begin{itemize}
    \item Einheitliche Formatierung und Versionierung der Dokumente.
    \item Verwendung von Templates zur Sicherstellung einer konsistenten Struktur.
    \item Regelmäßige Überprüfung und Aktualisierung der Dokumentation.
    \item Definierte Prozesse zur Nachverfolgbarkeit von Änderungen.
\end{itemize}
\newpage

\section{Umgang mit sich weiterentwickelnden Dokumenten}
\begin{itemize}
    \item \textbf{Versionierung}: Jede Dokumentenänderung wird mit einer neuen Versionsnummer versehen.
    \item \textbf{Historisierung}: Ältere Versionen bleiben zur Nachverfolgbarkeit erhalten.
    \item \textbf{Regelmäßige Reviews}: Überprüfung und Aktualisierung durch definierte Reviewer.
\end{itemize}
\newpage

\section{Fazit}
Dieser Qualitätssicherungsplan stellt sicher, dass die Qualitätsstandards des Projekts eingehalten werden. Durch klare Strukturen, definierte Prozesse und die Einhaltung bewährter Methoden wird die langfristige Wartbarkeit und Erweiterbarkeit des Codes sowie der gesamten Projektdokumentation gewährleistet.
\newpage

\bibliography{literatur}

\end{document}